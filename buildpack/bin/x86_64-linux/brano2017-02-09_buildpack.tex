\documentclass[11pt,a4paper]{article}
\usepackage[a4paper, total={8in, 9in}]{geometry}
\usepackage{graphicx}  % for images
\usepackage{caption}  % for subfigures
\usepackage{subcaption}  % for subfigures
\usepackage{chngcntr}
% \usepackage{filecontents}  % for base64 images
\graphicspath{ {/tmp}{/home} } % for images
\DeclareGraphicsExtensions{.pdf,.png,.jpg}
\usepackage{booktabs}
\usepackage{fontspec} % to use a custom font (e.g., Roboto). To use with lualatex instead of pdflatex
\newfontfamily\arabicfont[Path=../../texmf-dist/fonts/truetype/public/amiri/,
      Extension      = .ttf,
      UprightFont    = *-regular,
      ItalicFont     = *-slanted,
      BoldFont       = *-bold,
      BoldItalicFont = *-boldslanted,
      Script=Arabic,
      Scale=1.1,]{amiri}
\newfontfamily\englishfont[Path=../../texmf-dist/fonts/truetype/google/roboto/,
      Extension      = .ttf,
      UprightFont    = *-Light,
      ItalicFont     = *-LightItalic,
      BoldFont       = *-Bold,
      BoldItalicFont = *-BoldItalic,]{Roboto}
% \setmainfont[Path=/usr/share/texlive/texmf-dist/fonts/truetype/google/roboto/,
%      Extension      = .ttf,
%      UprightFont    = *-Light,
%      ItalicFont     = *-LightItalic,
%      BoldFont       = *-Bold,
%      BoldItalicFont = *-BoldItalic,
%      ]{Roboto}
\setmainfont[Path=../../texmf-dist/fonts/truetype/public/amiri/,
      Extension      = .ttf,
      UprightFont    = *-regular,
      ItalicFont     = *-slanted,
      BoldFont       = *-bold,
      BoldItalicFont = *-boldslanted,
      Script=Arabic,
      Scale=1.1,]{amiri}
\usepackage{lastpage} % to use the last page number in footer
\usepackage{datetime} % to get months as a words, not numbers
\usepackage[table,xcdraw]{xcolor} % for tables
\usepackage{multirow} % for tables
% \usepackage{array} % for tables
\usepackage{fancyhdr}
\pagestyle{fancy}  % to have header and footer

% \usepackage{grffile} % to manage fancy characters in figure file names (i.e., underscores)

\usepackage[hidelinks]{hyperref}  % to make clickable table of contents
% \hypersetup{colorlinks,allcolor=black}  % to set up link colors (probably unnecessary)

\usepackage{float}  % must be loaded before bidi package (loaded by polyglossia)

% polyglossia stuff:
\usepackage{polyglossia}
\setmainlanguage[numerals=maghrib]{arabic}
\setotherlanguage{english}


% the next 5 lines are to set the figure positioning. Figures in spare pages are centered to
% cover all the blank space. With these settings they are forced to align to the top of the page
\makeatletter
\setlength\@fptop{0pt}
\setlength\@fpsep{30pt plus 0fil}
\setlength\@fpbot{0pt}
\makeatother

\setlength{\abovetopsep}{1ex} % to set up the separation distance between the caption and the table

\setlength\topmargin{-40pt} % Top margin
\setlength\headheight{20pt} % Header height
%\setlength\textwidth{7.0in} % Text width
\setlength\textheight{9.5in} % Text height
%\setlength\oddsidemargin{-30pt} % Left margin
%\setlength\evensidemargin{-30pt} % Left margin (even pages) - only relevant with 'twoside' article option

\DeclareCaptionLabelFormat{arabic}{#2 #1}
\captionsetup{labelfont=it,textfont=bf,labelformat=arabic}
\appto\captionsarabic{\renewcommand{\figurename}{صورة}}
\captionsetup[table]{name=\textarabic{الطاولة}}
\counterwithin{figure}{section}

\renewcommand*\contentsname{\textarabic{جدول المحتويات}}




\begin{document}

\includegraphics{myfigure.png}

\newpage

\fancyhf{}
\fancyhead[R]{ test  }
\fancyhead[L]{\includegraphics[width=2cm]{Solargis-noR-RGB.png}} 
\fancyfoot[R]{\thepage / \pageref{LastPage}}
\fancyfoot[L]{Chapter \nouppercase{\leftmark} \ {\textcopyright} Solargis, {\monthname} \the\year}

\title{ Solargis PVplanner report }
%\author{\textenglish{Solargis\texttrademark  s.r.o}}
\author{Solargis\texttrademark  s.r.o}

\maketitle
\thispagestyle{empty}

\newpage
\setcounter{page}{1}

\tableofcontents
\listoffigures
\listoftables

\newpage

\section{ الموقع الجغرافي }

\begin{table}[h!]
% \centering
\begin{tabular} { p{3cm}p{5cm} }
 \textbf{\textarabic{  اسم الموقع الجغرافي }} &
\cellcolor[HTML]{ FFFFFF } Souissi

\\

 \textbf{\textarabic{  خط العرض }} &
\cellcolor[HTML]{ FFFFFF } 33.951904

\\

 \textbf{\textarabic{  خط الطول }} &
\cellcolor[HTML]{ FFFFFF } -6.804657

\\

\end{tabular}
\end{table}
\begin{figure}[h]
\centering
\includegraphics[width=0.5\textwidth]{overview_map.png}
\end{figure}

\section{ الإشعاع الكلي الأفقي: المجموع الشهري و السنوي }

\newpage

\begin{table}[h!]
% \centering

\begin{tabular} { llllllllllllll }
 \hline

\rowcolor[HTML]{ EFEFEF }
\textbf{  }

& \textbf{\textarabic{ يناير }}

& \textbf{\textarabic{ فبراير }}

& \textbf{\textarabic{ مارس }}

& \textbf{\textarabic{ أبريل }}

& \textbf{\textarabic{ ماي }}

& \textbf{\textarabic{ يونيو }}

& \textbf{\textarabic{ يوليوز }}

& \textbf{\textarabic{ غشت }}

& \textbf{\textarabic{ شتنبر }}

& \textbf{\textarabic{ أكتوبر }}

& \textbf{\textarabic{ نونبر }}

& \textbf{\textarabic{ دجنبر }}

& \textbf{  }
 \\  \hline

 2012 &

\cellcolor[HTML]{ E9F949 } 155
 & \cellcolor[HTML]{ E9F949 } 152  & \cellcolor[HTML]{ FA8C21 } 220  & \cellcolor[HTML]{ FA8C21 } 198  & \cellcolor[HTML]{ F2413D } 245  & \cellcolor[HTML]{ FFD62B } 188  & \cellcolor[HTML]{ 62B0C3 } 100  & \cellcolor[HTML]{ F2413D } 225  & \cellcolor[HTML]{ 77E16A } 122  & \cellcolor[HTML]{ FFD62B } 182  & \cellcolor[HTML]{ E9F949 } 160  & \cellcolor[HTML]{ 62B0C3 } 103

 & 2050

\\  \hline

 2013 &

\cellcolor[HTML]{ FFD62B } 176
 & \cellcolor[HTML]{ FA8C21 } 215  & \cellcolor[HTML]{ 62B0C3 } 107  & \cellcolor[HTML]{ 62B0C3 } 94  & \cellcolor[HTML]{ E9F949 } 149  & \cellcolor[HTML]{ 3A52bA } 62  & \cellcolor[HTML]{ E9F949 } 163  & \cellcolor[HTML]{ 3A52bA } 77  & \cellcolor[HTML]{ F2413D } 243  & \cellcolor[HTML]{ 3A52bA } 85  & \cellcolor[HTML]{ FA8C21 } 206  & \cellcolor[HTML]{ FA8C21 } 196

 & 1773

\\  \hline

 2014 &

\cellcolor[HTML]{ 62B0C3 } 100
 & \cellcolor[HTML]{ FA8C21 } 198  & \cellcolor[HTML]{ E9F949 } 151  & \cellcolor[HTML]{ 77E16A } 136  & \cellcolor[HTML]{ E9F949 } 149  & \cellcolor[HTML]{ F2413D } 230  & \cellcolor[HTML]{ F2413D } 236  & \cellcolor[HTML]{ 62B0C3 } 108  & \cellcolor[HTML]{ 77E16A } 117  & \cellcolor[HTML]{ FA8C21 } 204  & \cellcolor[HTML]{ E9F949 } 155  & \cellcolor[HTML]{ F2413D } 249

 & 2033

\\  \hline

 2015 &

\cellcolor[HTML]{ 62B0C3 } 93
 & \cellcolor[HTML]{ 62B0C3 } 90  & \cellcolor[HTML]{ 77E16A } 133  & \cellcolor[HTML]{ F2413D } 236  & \cellcolor[HTML]{ FA8C21 } 218  & \cellcolor[HTML]{ FFD62B } 184  & \cellcolor[HTML]{ FA8C21 } 217  & \cellcolor[HTML]{ 62B0C3 } 98  & \cellcolor[HTML]{ E9F949 } 144  & \cellcolor[HTML]{ 3A52bA } 64  & \cellcolor[HTML]{ FA8C21 } 206  & \cellcolor[HTML]{ FFD62B } 172

 & 1855

\\  \hline

 2016 &

\cellcolor[HTML]{ 62B0C3 } 103
 & \cellcolor[HTML]{ 3A52bA } 63  & \cellcolor[HTML]{ 3A52bA } 66  & \cellcolor[HTML]{ F2413D } 227  & \cellcolor[HTML]{ FFD62B } 194  & \cellcolor[HTML]{ E9F949 } 147  & \cellcolor[HTML]{ F2413D } 240  & \cellcolor[HTML]{ FA8C21 } 212  & \cellcolor[HTML]{ 3A52bA } 82  & \cellcolor[HTML]{ F2413D } 225  & \cellcolor[HTML]{ FA8C21 } 204  & \cellcolor[HTML]{ 77E16A } 136

 & 1899

\\  \hline
\rowcolor[HTML]{ EFEFEF }
\textbf{ LTA }
& \textbf{ 125 }
& \textbf{ 143 }
& \textbf{ 135 }
& \textbf{ 178 }
& \textbf{ 191 }
& \textbf{ 162 }
& \textbf{ 191 }
& \textbf{ 144 }
& \textbf{ 141 }
& \textbf{ 152 }
& \textbf{ 186 }
& \textbf{ 171 }
& \textbf{ 1922 }

\\  \hline
\end{tabular}

\caption[dcdcds]{
\textarabic{
LTA \_ او معدل المدى الطويل. تقلبات المعدل السنوي محددة بانحراف معياري واحد (ابيض: سنوات عادية, احمر او ازرق: سنة 
ذات حالة قصوى), الانحراف المعياري السنوي STDEV يساوي
}
}
\label{ LTA-_-او-معدل-المدى-الطويل.-تقلبات-المعدل-السنوي-محددة-ب
انحراف-معياري-واحد-(ابيض:-سنوات-عادية,-احمر-او-ازرق:-سنة-ذات-حالة-قصوى),-الانحراف-المعياري-السنوي-STDEV-يساوي- }

\end{table}
\section{ طريقة الاستخدام }

% \begin{Arabic}
البيانات الخاصة بموارد الطاقة الشمسية، بدرجة حرارة الهواء، بالطاقة 
الكهروضوئية المتوقعة (PVOUT) وكذا الخرائط المتوفرة في بوابة أطلس الموارد الشمسية 
للمغرب  يمكن استعمالها كمؤشر, يساعد على فهم إمكانيات الطاقة الشمسية في المغرب,
وتعتبر هذه البوابة أداة مفيدة لاختيار وتقييم أولي للمواقع المراد إنجازها. و يمثل
ال PVOUT تقدير الطاقة   المتوقع إنتاجها أثناء بدء  تشغيل  النظام الكهروضوئي PV. 
التقديرات دقيقة بما فيه الكفاية للأنظمة الكهروضوئية الصغيرة والمتوسطة الحجم وذات 
اللوحات الموجهة إلى الجنوب بالزاوية المثلى كما هو مبين في التطبيق الخاص بأطلس 
الموارد الشمسية للمغرب.   لتصميم وتمويل مشاريع كبيرة، لابد من التوفر على، أو طلب،
معلومات دقيقة إضافية، مثلا: 1. مواصفات دقيقة للنظام الكھروضوئی المراد استعماله, 2.
محاكاة الطاقة الكهروضوئية ارتكازا على بيانات السلسلة الزمنية التاريخية لعدة سنوات
أو TMY على الأقل, 3. اخذ بعين الاعتبار  حجم الدقة النسبية المتعلقة بالتقلبات 
البينية السنوية و حجم الدقة النسبية المتعلقة بال P90 للإنتاج الكهروضوئي, 4. 
كمية الطاقة المتوقع إنتاجها مع الأخذ بعين الاعتبار مدة الاستغلال ونسبة تدهور الأداء
الخاص بمكونات نظام النظام الكهروضوئي. ويمكن الاطلاع على مزيد من المعلومات حول تقييم إ
نتاج النظام الكهروضوئي الكامل على: https://solaratlas.masen.ma/about\#how\_to\_use.
% \end{Arabic}

\section{ تنبيه  و معلومات  قانونية }

\begin{Arabic}
نظرا لعدم التيقن التام من البيانات والعمليات الحسابية المستخدمة، فإن سولارجيس وكذا الوكالة 
المغربية للطاقة الشمسية لا يضمنون دقة التقديرات. كما يجب الإشارة إلى أنه قد تم القيام 
بالحد الأقصى الممكن لتقييم دقيق لمعلمات الطقس استنادا إلى أفضل البيانات المتاحة،
والبرمجيات والمعرفة المتوفرة. لذا فسولارجيس والوكالة المغربية للطاقة الشمسية غير مسؤولون 
عن أي خلل أو نتائج عرضية أو أضرار مباشرة أو غير مباشرة أو أضرار مزعومة، نتجت عن 
استخدام تقارير البيانات المستخرجة. حقوق التأليف والنشر لسولارجيس. جميع الحقوق محفوظة.
\end{Arabic}

\section{ الدقة النسبية  للبيانات }

\begin{Arabic}
الدقة النسبية
المتعلقة بالمخرجات السنوية  لتصميم سولارجيس الخاصة بمنطقة المغرب,
يمكن  تقديمها على النحو التالي: بالنسبة للإشعاع الكلي الافقي  ±4.0\%,
بالنسبة للإشعاع العمودي المباشر ±9.0\% ,بالنسبة للإشعاع الكلي المائل 
±5.0\%. أما بالنسبة لدرجة حرارة الهواء السنوية لجل المحطات الأرضية 
±1.0ºC وذلك بالرغم من أن درجة حرارة الهواء يمكن أن تصل إلى نسب عالية
مثل \_4.0ºC على أرضية ذات تضاريس معقدة، مثل سفوح جبال الأطلس الكبير. أما 
بخصوص بيانات الطاقة الكهروضوئية المتوقعة سنويا فإن الدقة النسبية قد تصل 
حوالي ±6.5\%. يمكن الاطلاع على مزيد من المعلومات حول الدقة النسبية المتعلقة 
بالبيانات على الرابط: https://solaratlas.masen.ma/about\#solargis\_data.
\end{Arabic}

\section{ Minipage test }

\begin{minipage}{0.45\textwidth}
\begin{figure}[H]
\centering
\includegraphics[width=0.5\textwidth]{overview_map.png}
\caption{Location}  % THIS DOESN'T WORK WELL!
\end{figure}
\end{minipage}
\begin{minipage}{0.45\textwidth}
\begin{figure}[H]
\centering
\includegraphics[width=0.5\textwidth]{overview_map.png}
\caption{Location}  % THIS DOESN'T WORK WELL!
\end{figure}
\end{minipage}

\begin{minipage}{0.45\textwidth}
\begin{table}[H]
% \centering
\begin{tabular} { p{3cm}p{5cm} }
 \textbf{ Location name: } & 
\cellcolor[HTML]{ FFFFFF } Souissi

\\ 

 \textbf{ Latitude: } & 
\cellcolor[HTML]{ FFFFFF } 33.951904

\\ 

 \textbf{ Longitude: } & 
\cellcolor[HTML]{ FFFFFF } -6.804657

\\ 

\end{tabular}
\end{table}
\end{minipage}
\begin{minipage}{0.45\textwidth}
    \centering
    \begin{tabular}{cc}\hline
      Table head & Table head \\ \hline
        Some values & Some values \\
        Some values & Some values \\
        Some values & Some values \\
        Some values & Some values \\
        Some values & Some values \\
        Some values & Some values \\
        Some values & Some values \\
        Some values & Some values \\
        Some values & Some values \\
        Some values & Some values \\
        Some values & Some values \\
        Some values & Some values \\
        Some values & Some values \\
        Some values & Some values \\
        Some values & Some values \\
        Some values & Some values \\
        Some values & Some values \\
        Some values & Some values \\
        Some values & Some values \\
        Some values & Some values \\
        Some values & Some values \\
        Some values & Some values \\
        Some values & Some values \\
        Some values & Some values \\
        Some values & Some values \\
        Some values & Some values \\
        Some values & Some values \\
        Some values & Some values \\
        Some values & Some values \\
        Some values & Some values \\
        Some values & Some values \\
        Some values & Some values \\
        Some values & Some values \\
        Some values & Some values \\
        Some values & Some values \\
        Some values & Some values \\
        Some values & Some values \\
        Some values & Some values \\
        Some values & Some values \\
        Some values & Some values \\
        Some values & Some values \\
        Some values & Some values \\
        Some values & Some values \\
        Some values & Some values \\
        Some values & Some values \\
        Some values & Some values \\
        Some values & Some values \\
        Some values & Some values \\
        Some values & Some values \\
        Some values & Some values \\
        Some values & Some values \\
        Some values & Some values \\
        Some values & Some values \\
        Some values & Some values \\ \hline
      \end{tabular}
      \captionof{table}{A table beside a figure}
\end{minipage}


\end{document}


% rsync -a /media/sf_tmp__/buildpack /home/m/ ; chown -Rh m:m /home/m/buildpack ;
% export PATH=$PATH:/home/m/buildpack/bin/x86_64-linux ; cd /home/m/buildpack/bin/x86_64-linux; mktexlsr
% xelatex --shell-escape -synctex=1 -interaction=nonstopmode *.tex

